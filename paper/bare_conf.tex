% TUAC / IEEE konferans bildirisi taslağına göre hazırlanmış makale
\documentclass[conference,a4paper]{IEEEtran}
\IEEEoverridecommandlockouts

\usepackage[turkish]{babel}
\usepackage[utf8]{inputenc}
\usepackage[T1]{fontenc}
\usepackage[pdftex]{graphicx}
\usepackage{multirow}
\usepackage{cite}
\usepackage[cmex10]{amsmath}
\usepackage{siunitx}
\usepackage{array}
\usepackage[caption=false,lofdepth,lotdepth]{subfig}
\usepackage{acronym}
\usepackage{xcolor}
\usepackage{microtype}

\hyphenation{op-tical net-works semi-conduc-tor}
\setlength{\textfloatsep}{5pt}

\AtBeginDocument{\renewcommand\tablename{TABLO}}

% Türkçe özet ve anahtar kelime ortamlarını IEEE ortamlarına eşleştir
\renewenvironment{ozet}{%
  \renewcommand\abstractname{Özet}%
  \begin{abstract}%
}{%
  \end{abstract}%
}
\renewenvironment{IEEEanahtar}{%
  \renewcommand\IEEEkeywordsname{Anahtar Kelimeler}%
  \begin{IEEEkeywords}%
}{%
  \end{IEEEkeywords}%
}

\begin{document}

\title{Model Bağlam Protokolü (MCP) Ekosisteminin Eleştirel Bir Güvenlik İncelemesi\\
       A Critical Security Review of the Model Context Protocol (MCP) Ecosystem}

\author{%
  \IEEEauthorblockN{Yusuf Talha ARABACI}%
  \IEEEauthorblockA{%
    Yazılım Mühendisliği Yüksek Lisans Öğrencisi\\%
    Karabük Üniversitesi\\%
    Karabük, Türkiye%
  }%
}

\maketitle

\begin{ozet}
Model Bağlam Protokolü (Model Context Protocol, MCP), büyük dil modelleri (LLM) ile harici araçlar ve veri kaynakları arasında çift yönlü, şema odaklı iletişim ve dinamik keşif sağlayan açık bir standarttır. Bu standart, entegrasyon parçalanmasını azaltarak ekosistemi sadeleştirirken; araç zehirleme, prompt enjeksiyonu, yanlış yapılandırılmış veya açığa çıkmış sunucular ve zayıf tedarik zinciri hijyeni gibi yeni güvenlik risklerini de beraberinde getirmektedir. Bu çalışma, MCP ekosisteminin mimarisini, Hou vd.\ ve Hasan vd.\ tarafından önerilen tehdit taksonomisini, 1\,899 MCP sunucusunu kapsayan ampirik bulguları, Song vd., Luo vd., Yin vd.\ ve Fan vd.\ tarafından sunulan benchmark sonuçlarını ve Xing vd.\ ile He vd.\ tarafından geliştirilen savunma ve kırmızı takım çerçevelerini bir araya getirerek eleştirel bir güvenlik incelemesi sunmaktadır. Ayrıca ilgili protokoller ve alan uygulamaları üzerinden MCP'nin gelecekteki araştırma gündemi tartışılmaktadır.
\end{ozet}
\begin{IEEEanahtar}
Model Bağlam Protokolü, MCP, güvenlik, araç zehirleme, benchmark, kırmızı takım, tedarik zinciri.
\end{IEEEanahtar}

\begin{abstract}
The Model Context Protocol (MCP) is an open standard that enables schema-driven, bidirectional communication and dynamic discovery between large language models (LLMs) and external tools or resources. While it significantly reduces integration fragmentation, it also introduces new security risks such as tool poisoning, prompt injection, exposed or misconfigured servers, and weak software supply-chain hygiene. This paper provides a critical security review of the MCP ecosystem by synthesizing its architecture, the threat taxonomy proposed by Hou et al.\ and Hasan et al., empirical findings from 1,899 MCP servers, benchmark results reported by Song et al., Luo et al., Yin et al., and Fan et al., and defense as well as red-teaming frameworks introduced by Xing et al.\ and He et al. We further discuss related protocols and domain-specific applications to highlight open research directions for securing MCP-based systems.
\end{abstract}
\begin{IEEEkeywords}
Model Context Protocol, MCP, security, tool poisoning, benchmarks, red teaming, supply chain security.
\end{IEEEkeywords}

\IEEEpeerreviewmaketitle

\section{Giriş}
Model Bağlam Protokolü (MCP), büyük dil modellerinin yalnızca metin üreten sistemler olmaktan çıkarak gerçek dünya sistemleriyle etkileşime giren otonom ajanlar hâline gelmesini mümkün kılan temel yapı taşlarından biridir. MCP, LLM barındıran istemci (host) ile harici işlevleri ve veri kaynaklarını sunan sunucular arasında JSON--RPC 2.0 tabanlı bir istemci--sunucu iletişimi tanımlar ve araçların (tools) ile veri kaynaklarının (resources) şema temelli olarak keşfedilmesini sağlar \cite{hou2025landscape,singh2025survey}. 

Bu mimari, entegrasyon maliyetlerini düşürme ve birlikte çalışabilirliği artırma açısından önemli avantajlar sunarken; araç açıklamalarına duyulan güven, dinamik keşif ve karmaşık yürütme akışları nedeniyle yeni bir saldırı yüzeyi de ortaya çıkarmaktadır. Literatürdeki son çalışmalar, MCP ekosistemini mimari ve protokol tasarımı, tehdit modelleme, ampirik güvenlik taramaları, benchmark'lar, savunma çerçeveleri ve alan uygulamaları üzerinden kapsamlı biçimde incelemiştir \cite{hou2025landscape,hasan2025firstglance,xing2025guard,song2025help,luo2025universe,yin2025livemcp,fan2025mcptoolbench}.

Bu bildirinin amacı, söz konusu literatürü bir araya getirerek MCP ekosistemine yönelik eleştirel ve bütüncül bir güvenlik değerlendirmesi sunmak, zayıf noktaları ve açık araştırma alanlarını ortaya koymaktır. Çalışma, önce MCP mimarisini ve temel bileşenlerini özetlemekte, ardından tehdit modeli ve saldırı vektörlerini tartışmakta; ampirik bulgular ve benchmark sonuçlarını değerlendirmekte; savunma stratejileri ve iyi uygulamalar ile ilgili protokoller ve alan uygulamalarını incelemekte ve son olarak geleceğe dönük araştırma önerileriyle sonuçlanmaktadır.

\section{Protokol Mimarisinin Güvenlik Etkileri}
MCP, LLM merkezli uygulamalarda araç kullanımını standartlaştırmak için tasarlanan katmanlı bir mimariye sahiptir. Temel bileşenler aşağıdaki şekilde özetlenebilir:
\begin{itemize}
  \item \textbf{Host / İstemci:} LLM'yi barındıran uygulama (örneğin IDE, sohbet istemcisi) ve MCP sunucularıyla JSON--RPC üzerinden konuşan istemci bileşenidir. Araç ve kaynak yeteneklerini keşfeder, LLM'ye iletir ve çağrıları yürütür.
  \item \textbf{Sunucu:} Harici işlevleri (araçlar) ve veri kaynaklarını standartlaştırılmış biçimde sunan bağımsız süreçtir. Dosya sistemi, veritabanı ve üçüncü taraf API'ler gibi sistemlere erişir.
  \item \textbf{Araçlar ve Kaynaklar:} LLM'nin çağırabildiği yürütülebilir işlevler (tools) ve okuyabildiği veri kaynaklarıdır (resources). Açıklama (description) ve şema alanları hem planlama için kritik hem de saldırı yüzeyi açısından hassastır.
  \item \textbf{Taşıma Katmanı:} Host ile sunucu arasında StdIO ya da HTTPS/SSE üzerinden iletişim kurulur ve JSON--RPC 2.0 mesajları taşınır. Şifreleme (TLS) ve kimlik doğrulama (örneğin OAuth~2.1, mTLS) kalitesi saldırı yüzeyini doğrudan etkiler.
\end{itemize}

% Şekil 1: MCP mimarisinin şematik gösterimi
\begin{figure}[!t]
  \centering
  \resizebox{\columnwidth}{!}{%
    \includegraphics{images/protocol}%
  }
  \caption{MCP mimarisinde host, sunucu, araçlar ve kaynaklar arasındaki etkileşim ve temel bileşenler.}
  \label{fig:protocol-architecture}
\end{figure}

% Şekil 2: MCP istemci–sunucu mimarisinin temel akışı
\begin{figure}[!t]
  \centering
  \resizebox{\columnwidth}{!}{%
    \includegraphics{images/mimari}%
  }
  \caption{MCP istemci--sunucu mimarisinin temel akışı. Şema, LLM'nin akıl yürütme ortamı ile harici yürütme ortamı (araçlar ve kaynaklar) arasında JSON--RPC tabanlı iletişim üzerinden kurulan standart güven sınırını ve N$\times$M entegrasyon problemi için getirilen mimari çözümü göstermektedir.}
  \label{fig:mcp-flow}
\end{figure}

Hou vd.\ \cite{hou2025landscape}, MCP sunucularının yaşam döngüsünü \emph{oluşturma}, \emph{dağıtım}, \emph{ișletim} ve \emph{bakım} olmak üzere dört fazda ele almakta ve her fazda ortaya çıkabilecek on altı tehdit senaryosu tanımlamaktadır. Hasan vd.\ \cite{hasan2025firstglance} ise 1\,899 açık kaynak MCP sunucusunda yaptıkları taramada, kimlik bilgisi sızıntıları, hatalı erişim kontrolü ve kötü yapılandırılmış araç tanımları gibi MCP'ye özgü risklerin, klasik web ve API güvenliği sorunlarıyla birleşerek daha karmaşık saldırı zincirlerine yol açabildiğini göstermektedir.

\section{Tehdit Modeli ve Saldırı Vektörleri}
MCP ekosistemindeki tehditler, literatürde genellikle şu dört aktör tipi altında toplanmaktadır: (i) kötü niyetli geliştiriciler, (ii) dış saldırganlar, (iii) kötü niyetli son kullanıcılar ve (iv) yazılım ile yapılandırma hatalarından kaynaklanan zafiyetler \cite{hou2025landscape,hasan2025firstglance}.

\subsection{Kötü Niyetli Geliştirici Kaynaklı Tehditler}
Araç zehirleme (tool poisoning), MCP'ye özgü en kritik tehditlerden biridir. Geliştirici, aracın açıklama alanına gizli talimatlar veya yanıltıcı şemalar yerleştirerek LLM'nin planlama sürecini manipüle edebilir; bu durum hassas dosyaların okunması, gizli anahtarların sızdırılması veya beklenmedik yüksek etkili çağrıların tetiklenmesiyle sonuçlanabilir. Aynı şekilde, sahte veya gölge sunucularla isim çakışması yaratmak, LLM'nin yanlış sunucuyu seçmesine ve meşru görünen ancak zararlı araçları çağırmasına yol açabilir.

\subsection{Dış Saldırganlar ve Kötü Niyetli Kullanıcılar}
Dolaylı prompt enjeksiyonu, LLM'nin bir MCP aracı üzerinden okuyacağı harici kaynağa (örneğin bir RAG belgesi, hata kaydı veya web içeriği) zararlı talimatların gömülmesiyle gerçekleşir. Model, bu içeriği güvenilir bağlam gibi yorumlayarak kritik araç çağrılarını yanlış yönlendirebilir. İnternete gereksiz biçimde açılmış veya zayıf kimlik doğrulamalı MCP sunucuları, doğrudan veri sızıntısı ya da sistem manipülasyonuna kapı aralar.

Kötü niyetli kullanıcılar açısından dikkat çekici vektörlerden biri, zincirleme araç kötüye kullanımıdır (Sequential Tool Attack Chaining -- STAC). Tek tek düşük riskli görünen araçların ardışık kullanımı, toplu olarak yüksek etkili operasyonlara (örneğin yapılandırma dosyalarının okunması, filtrelenmesi ve harici ortamlara aktarılması) dönüşebilir. MCP, protokol düzeyinde tam izolasyon sağlamadığından; gerçek güvenlik sınırları barındırma ortamındaki sandbox, yetkilendirme ve politika katmanlarıyla çizilir.

\subsection{Yazılım ve Yapılandırma Hataları}
Ampirik çalışmalar, kimlik bilgisi sızıntısı (yapılandırma dosyalarında düz metin API anahtarları), komut enjeksiyonu (filtrelenmemiş girdiden kabuk çağrıları), eksik TLS/OAuth yapılandırmaları ve yetersiz giriş doğrulama gibi zafiyetlerin MCP sunucularında yaygın olduğunu göstermektedir \cite{hasan2025firstglance}. Bu zafiyetler klasik web ve API güvenliğiyle aynı kökten gelse de, LLM'lerin otomatik araç çağrılarıyla birleştiğinde etkileri büyümektedir.

\section{Ampirik Bulgular ve Benchmark Sonuçları}
Hasan vd.\ \cite{hasan2025firstglance}, 1\,899 açık kaynak MCP sunucusunu inceleyerek sekiz MCP-özel zafiyet sınıfı tanımlamış; genel güvenlik açığı oranını yaklaşık \%7, araç zehirleme riskini ise \%5 civarında raporlamıştır. Kod kokuları ve bakım sorunları da göz önüne alındığında, ekosistemin bütününde anlamlı bir risk yoğunluğu olduğu görülmektedir.

Performans ve dayanıklılık açısından, MCP'li ajanların davranışı çeşitli benchmark çalışmalarıyla değerlendirilmiştir. MCPToolBench++ \cite{fan2025mcptoolbench}, çok sayıda gerçekçi aracı kapsayan geniş ölçekli bir benchmark sunarken; MCP-Universe \cite{luo2025universe} gerçek MCP sunucularıyla etkileşim kuran görevler üzerinden LLM'lerin planlama ve hata toleransını ölçmektedir. LiveMCP-101 \cite{yin2025livemcp} çalışması ise zorlu sorgular altında MCP destekli ajanların stres testine odaklanmakta ve protokol seviyesindeki tasarım kararlarının hata modlarına nasıl yansıdığını ortaya koymaktadır.

Song vd.\ \cite{song2025help}, MCP entegrasyonunun her zaman performans artışı sağlamadığını; bazı görevlerde bağlam yönetimi, protokol karmaşıklığı ve araç gecikmeleri nedeniyle toplam sistem verimliliğinin düştüğünü göstermektedir. Bu bulgular, MCP'nin sadece fonksiyonel değil, aynı zamanda maliyet ve risk boyutlarıyla da değerlendirilmesi gerektiğine işaret etmektedir.

% Şekil 3: Benchmark ve stres testlerinden elde edilen bulguların özeti
\begin{figure}[!t]
  \centering
  \resizebox{\columnwidth}{!}{%
    \includegraphics{images/diagram}%
  }
  \caption{Benchmark ve stres testlerinden elde edilen bulguların, güvenlik ve performans boyutlarıyla birlikte şematik özeti.}
  \label{fig:benchmarks}
\end{figure}

\section{Savunma Stratejileri ve İyi Uygulamalar}
Xing vd.\ tarafından önerilen MCP-Guard çerçevesi \cite{xing2025guard}, protokol bütünlüğünü korumak için politika tabanlı doğrulama, davranışsal günlükleme ve anomali tespiti bileşenlerini bir araya getirmektedir. Bu çerçeve, araç açıklamalarını statik olarak analiz etmekle kalmayıp, çalışma zamanında gerçekleşen araç çağrılarını da politikalarla karşılaştırarak ihlalleri tespit etmeye odaklanır.

He vd.\ \cite{he2025automated} ise MCP araçlarını kullanabilen LLM tabanlı ajanlara yönelik otomatik kırmızı takım (red team) teknikleri geliştirmekte; zehirlenmiş araç açıklamaları, zincirleme saldırı senaryoları ve hassas veri sızıntılarını tetikleyen girişler üretmektedir. Bu tür çalışmalar, MCP ekosisteminde güvenlik değerlendirmesinin manuel testlerin ötesine geçmesi gerektiğini göstermektedir.

Uygulama geliştiricileri için iyi uygulamalar şu başlıklarda özetlenebilir:
\begin{itemize}
  \item Araç açıklamalarında en az ayrıcalık ve açık, doğrulanabilir şemalar kullanmak.
  \item Sunucuları gereksiz yere internete açmamak; güçlü kimlik doğrulama ve yetkilendirme katmanları uygulamak.
  \item Tedarik zinciri hijyenine dikkat etmek; MCP sunucularını ve bağımlılıklarını düzenli olarak taramak ve güncellemek.
  \item Plan tabanlı testler ve otomatik kırmızı takım senaryolarıyla LLM-araç etkileşimini doğrulamak.
\end{itemize}

\section{İlgili Protokoller ve Karşılaştırma}
MCP, ajan ekosisteminde önerilen tek standart değildir. Ehtesham vd.\ \cite{ehtesham2025survey}, Model Context Protocol (MCP), Agent Communication Protocol (ACP), Agent-to-Agent Protocol (A2A) ve Agent Network Protocol (ANP) gibi protokolleri bir arada ele alarak birlikte çalışabilirlik ve güvenlik özellikleri açısından karşılaştırmaktadır. MCP, özellikle LLM odaklı araç ve kaynak entegrasyonunu hedeflerken; ACP ve A2A daha çok ajanlar arası mesajlaşma semantiğine, ANP ise ağ düzeyinde çok ajanlı orkestrasyona odaklanmaktadır.

Krishnan \cite{krishnan2025multiagent}, MCP'nin çok ajanlı sistem mimarilerinde kullanımı üzerine ayrıntılı bir inceleme sunmakta ve protokolün orkestrasyon, gözlemlenebilirlik ve hata toleransı açısından sağladığı avantajları tartışmaktadır. Bu karşılaştırmalar, MCP'nin yalnızca teknik bir entegrasyon standardı değil, aynı zamanda çok ajanlı sistemler için stratejik bir tasarım kararı olduğunu göstermektedir.

\section{Alan Uygulamaları ve Endüstri Bağlamı}
MCP, farklı alanlarda önerilen uygulamalarla giderek daha geniş bir kullanım alanı bulmaktadır. MCPmed \cite{flotho2025mcpmed}, biyoinformatik web servislerinin MCP üzerinden LLM'lerle bütünleştirilmesini savunarak, bilimsel keşif süreçlerinde yeniden üretilebilirlik ve şeffaflığa vurgu yapmaktadır. Chhetri vd.\ \cite{chhetri2025transport} ise uyarlanabilir ulaşım sistemlerinde MCP benzeri protokollerin kullanılmasını tartışmakta ve güvenlik ile emniyet gereksinimlerinin birlikte ele alınması gerektiğini öne sürmektedir.

AgentX \cite{tokal2025agentx} çalışması, FaaS tabanlı MCP servisleriyle dayanıklı ajan iş akışlarının orkestrasyonuna odaklanırken; Bhandarwar \cite{bhandarwar2025integrating} generatif yapay zekâ, MCP ve uygulamalı makine öğrenmesinin bütünleşik kullanımını incelemektedir. Kritik altyapılarda varlık keşfi \cite{coppolino2025asset} ve iktisadi araştırma için ajan tabanlı yaklaşımlar \cite{korinek2025agents} gibi çalışmalar, MCP'nin yalnızca teknik bir standart değil, aynı zamanda alanlararası bir güvenlik ve yönetişim problemi olduğunu göstermektedir.

\section{Sonuç}
MCP, birlikte çalışabilirlik ve araç keşfiyle önemli değer yaratırken, protokole özgü riskler de getirmektedir. Ampirik veriler anlamlı güvenlik açığı oranlarını; benchmark çalışmaları orkestrasyon kırılganlığını ve otomatik zehirleme riskini ortaya koymaktadır. Güvenli benimseme için katmanlı savunma, tedarik zinciri hijyeni, plan tabanlı testler ve kırmızı takım çalışmaları birlikte ele alınmalıdır. 

Bu çalışma, mevcut literatürü sentezleyerek MCP ekosisteminin güçlü ve zayıf yönlerini ortaya koymuş, ilgili protokoller ve alan uygulamaları üzerinden gelecekteki araştırma gündemini tartışmıştır. Önümüzdeki dönemde standartlaşma, denetim çerçeveleri, alan-özgü güvenlik politikaları ve otomatik güvenlik testi araçlarının MCP ekosistemine entegre edilmesi, protokolün sürdürülebilir ve güvenli biçimde yaygınlaşması için kritik görünmektedir.

\begin{thebibliography}{99}
\bibitem{hou2025landscape} X.~Hou, Y.~Zhao, S.~Wang, and H.~Wang, ``Model Context Protocol (MCP): Landscape, Security Threats, and Future Research Directions,'' \emph{arXiv preprint arXiv:2503.23278}, 2025.
\bibitem{krishnan2025multiagent} N.~Krishnan, ``Advancing Multi-Agent Systems Through Model Context Protocol: Architecture, Implementation, and Applications,'' \emph{arXiv preprint arXiv:2504.21030}, 2025.
\bibitem{ehtesham2025survey} A.~Ehtesham, A.~Singh, G.~K.~Gupta, and S.~Kumar, ``A survey of agent interoperability protocols: Model Context Protocol (MCP), Agent Communication Protocol (ACP), Agent-to-Agent Protocol (A2A), and Agent Network Protocol (ANP),'' \emph{arXiv preprint arXiv:2505.02279}, 2025.
\bibitem{hasan2025firstglance} M.~M.~Hasan, H.~Li, E.~Fallahzadeh, G.~K.~Rajbahadur, B.~Adams, and A.~E.~Hassan, ``Model Context Protocol (MCP) at First Glance: Studying the Security and Maintainability of MCP Servers,'' \emph{arXiv preprint arXiv:2506.13538}, 2025.
\bibitem{flotho2025mcpmed} M.~Flotho, I.~F.~Diks, P.~Flotho, L.-A.~G.~Molano, P.~Hirsch, and A.~Keller, ``MCPmed: A Call for MCP-Enabled Bioinformatics Web Services for LLM-Driven Discovery,'' \emph{arXiv preprint arXiv:2507.08055}, 2025.
\bibitem{mastouri2025rest} M.~Mastouri, E.~Ksontini, and W.~Kessentini, ``Making REST APIs Agent-Ready: From OpenAPI to MCP Servers for Tool-Augmented LLMs,'' \emph{arXiv preprint arXiv:2507.16044}, 2025.
\bibitem{fan2025mcptoolbench} S.~Fan, X.~Ding, L.~Zhang, and L.~Mo, ``MCPToolBench++: A Large Scale AI Agent Model Context Protocol MCP Tool Use Benchmark,'' \emph{arXiv preprint arXiv:2508.07575}, 2025.
\bibitem{xing2025guard} W.~Xing, Z.~Qi, Y.~Qin, Y.~Li, C.~Chang, J.~Yu, C.~Lin, Z.~Xie, and M.~Han, ``MCP-Guard: A Defense Framework for Model Context Protocol Integrity in Large Language Model Applications,'' \emph{arXiv preprint arXiv:2508.10991}, 2025.
\bibitem{song2025help} W.~Song, H.~Zhong, Z.~Ding, J.~Xue, and Y.~Li, ``Help or Hurdle? Rethinking Model Context Protocol-Augmented Large Language Models,'' \emph{arXiv preprint arXiv:2508.12566}, 2025.
\bibitem{luo2025universe} Z.~Luo, Z.~Shen, W.~Yang, Z.~Zhao, P.~Jwalapuram \emph{et al.}, ``MCP-Universe: Benchmarking Large Language Models with Real-World Model Context Protocol Servers,'' \emph{arXiv preprint arXiv:2508.14704}, 2025.
\bibitem{yin2025livemcp} M.~Yin, D.~Shen, S.~Xu, J.~Han, S.~Dong \emph{et al.}, ``LiveMCP-101: Stress Testing and Diagnosing MCP-enabled Agents on Challenging Queries,'' \emph{arXiv preprint arXiv:2508.15760}, 2025.
\bibitem{chhetri2025transport} G.~Chhetri, S.~Somvanshi, M.~M.~Islam, S.~Brotee, M.~S.~Mimi \emph{et al.}, ``Model Context Protocols in Adaptive Transport Systems: A Survey,'' \emph{arXiv preprint arXiv:2508.19239}, 2025.
\bibitem{tokal2025agentx} S.~S.~K.~A.~Tokal, V.~Jha, A.~Eswaran, P.~Jayachandran, and Y.~Simmhan, ``AgentX: Towards Orchestrating Robust Agentic Workflow Patterns with FaaS-hosted MCP Services,'' \emph{arXiv preprint arXiv:2509.07595}, 2025.
\bibitem{he2025automated} P.~He, C.~Li, B.~Zhao, T.~Du, and S.~Ji, ``Automatic Red Teaming LLM-based Agents with Model Context Protocol Tools,'' \emph{arXiv preprint arXiv:2509.21011}, 2025.
\bibitem{singh2025survey} A.~Singh, A.~Ehtesham, S.~Kumar, and T.~T.~Khoei, ``A Survey of the Model Context Protocol (MCP): Standardizing Context to Enhance Large Language Models (LLMs),'' \emph{Preprints 202504.0245}, 2025.
\bibitem{bhandarwar2025integrating} N.~Bhandarwar, ``Integrating Generative AI and Model Context Protocol (MCP) with Applied Machine Learning for Advanced Agentic AI Systems,'' \emph{International Journal of Computer Trends and Technology}, 2025.
\bibitem{coppolino2025asset} L.~Coppolino, A.~Iannaccone, R.~Nardone, and A.~Petruolo, ``Asset Discovery in Critical Infrastructures: An LLM-Based Approach,'' \emph{Electronics}, vol.~14, no.~32, p.~3267, 2025.
\bibitem{korinek2025agents} A.~Korinek, ``AI Agents for Economic Research,'' NBER Working Paper~34202, 2025.
\end{thebibliography}

\end{document}
